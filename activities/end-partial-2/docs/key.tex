\documentclass[../main.tex]{subfiles}

\begin{document}

\section{Exponential equations}

\begin{comment}
\begin{enumerate}
    \item $8e^{2x} = 20$
    \item $e^{3-2x} = 4$
    \item $e^{2x}-e^{x}-6=0$
    \item $3xe^x + x^2e^x = 0 $
\end{enumerate}
\end{comment}

\begin{enumerate}
    \item $x = 1/2 \ln(5/2) = \ln(\sqrt{5/2})$
    \item $x = 1/2 (3-\ln(4))$
    \item $e^{2x}-e^{x}-6=0$ % Procedure
    \item $3xe^x + x^2e^x = 0 $ % Procedure
\end{enumerate}


\begin{align*}
    e^{2x}-e^x-6 &= 0 \\
    \qty(e^{x})^{2}-e^x-6 &= 0 \\
    \mathrm{We~make~the~following~change~of~variable} & \alpha = e^x \\
    \qty(\alpha)^2-\alpha-6 &= 0 \\
    \alpha &= \frac{-(-1)\pm\sqrt{(-1)^2-4(1)(-6)}}{2(1)} \\
    \alpha &= \frac{1\pm\sqrt{1+24}}{2} \\
\end{align*}


\section{Logarithmic equations}

\begin{enumerate}
    \begin{minipage}[c]{0.45\textwidth}
        \item $4+3\log(2x)=16$
        \item $\log(x+2) + \log(x-1) = 1$
    \end{minipage}
    \begin{minipage}[c]{0.45\textwidth}
        \item $\ln(x) = 10$
        \item $\log_2(x) + \log_2(x-3) = 1$
    \end{minipage}
\end{enumerate}



\section{Exponential and logarithmic equations}
From the following equations, solve for $x$,
\begin{enumerate}
    \begin{minipage}[c]{0.45\textwidth}
        \item $ e^{3x-2} = e^{x^2} $
        \item $ 3\log(x) = 6 -2x $
        \item $ \log_3(x-8) + \log_3(x) = 2 $
        \item $ \log_2(x+a) + \log_2\qty(x-b) = c $
    \end{minipage}
    \begin{minipage}[c]{0.45\textwidth}
        \item $ 5^{x/10}+1 = 7 $
        \item $ 4-x^2 = e^{-2x} $
        \item $ \ln(x-2) + \ln(3) = \ln(5x-7) $
        \item $\qty(1-e^{t/\alpha})^{x} = \frac{a}{b} $
    \end{minipage}
\end{enumerate}




\section{Modeling with exponential functions}

The element Polonium-210 ($^{210}\mathrm{Po}$) has a half-life of \num{140} days, that is, $m(140) = m_o/2$.
Suppose a sample of this element has a mass of \SI{300}{\milli\gram}.
\begin{enumerate}[label=(\alph*)]
    \item Find a function that models the mass remaining after $t$ hours.
    \item Find te mass remaining after \num{300} days. 
    \item How long will it take the sample to decay to a mass of \SI{75}{\milli\gram}?
\end{enumerate}



\section{Modeling with logarithmic functions}

\paragraph{a)} The hydrogen ion concentration of a sample of each substance is given.
Calculate the pH of the substance,
\begin{enumerate}
    \item Lemon juice: $\mathrm{H}^+=\num{5.0d-3}$
    \item Tomato juice: $\mathrm{H}^+=\num{3.2d-4}$
    \item Seawater: $\mathrm{H}^+=\num{5.0d-9}$
\end{enumerate}

\paragraph{b)} the pH reading of a sample of each substance is given.
Calculate the hydrogen concentration of the substance,
\begin{enumerate}
    \item Vinegar: pH = \num{3.0}
    \item Milk: pH = \num{6.5}
\end{enumerate}

\section{Critical points}

Find the critical points of the following functions,
\begin{enumerate}
    \begin{minipage}[c]{0.45\textwidth}
        \item $f(x)=\num{5d3}e^{0.5 x}$
        \item $g(x)=-\log\qty[8x]$
        \item $h(x)=500e^{-0.2 x + 3}+150$
    \end{minipage}
    \begin{minipage}[c]{0.45\textwidth}
        \item $m(x)=\log\qty[\num{1d3}e^{10 x}+100]$
        \item $n(x)=3(x-3)e^{-5(x-3)}$
        \item $p(x)=3(x-5)\log\qty[-1/2(x-5)]$
    \end{minipage}
\end{enumerate}



\end{document}
