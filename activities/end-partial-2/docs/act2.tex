\documentclass[../main.tex]{subfiles}

\begin{document}

\section{Logarithmic equations}

\subsection{Examples}

\paragraph{1} To solve the following equation \[\log\qty(x^2+1)=\log\qty(x-2) + \log\qty(x+3)\] 
we are going to reduce the right hand side of the equation using properties of logarithms,
\begin{align*}
    \log\qty(x^2+1) &= \log\qty(x-2) + \log\qty(x+3) \\
    \log\qty(x^2+1) &= \log\qty[(x-2)(x+3)].
\end{align*}
Now, we are going to use the fact that the logarithm is a bijective function.
Therefore, to solve that equation, the arguments needs to be equal (as the example 1 of exponential functions).
Hence, we get the following polinomial equation,
\begin{align*}
    x^2+1 &= (x-2)(x+3) \\
    x^2+1 &= x^2\cancelto{x}{-2x+3x}-6 \\
    \cancelto{0}{x^2-x^2}-x &= \cancelto{-7}{-6-1} \\
-x &= -7 \\
    x &= 7.
\end{align*}

If we evaluate the equation at $x=7$ we can reassure that is the solution of the equation.

\paragraph{2} Now, we are going to solve the following equation $\log_2(25-x)=3$.
For this equation we can not use the same argument of example 1, because there is no logarithm function in both sides.
Hence we are going to apply the property of inverse functions, $a^{\log_a(x)}=x$ and $\log_a(a^x)=x$.

First we are going to apply the following change of variable $\alpha=25-x$, allowing us to re-write the equation as follows,
\begin{gather*}
    \log_2(25-x)=3\rightarrow\log_2(\alpha)=3.
\end{gather*}
Analyzing the equation we know that is a logartihm base $2$, so in order to use the inverse function property we need to apply an exponential function base $2$ in both sides,
\begin{align*}
    \cancelto{\alpha}{2^{\log_2(\alpha)}} &= 2^3 \\
    \alpha &= 2^3,
\end{align*}
now, we undo the change of variable and apply algebra tools to solve for $x$,
\begin{align*}
    25-x &= 2^3 \\
    -x &= 2^3 - 25 \\
    x &= \cancelto{17}{25 - 2^3} \\
      &= 17.
\end{align*}

Looking at that procedure we can get a general procedure:
\paragraph{Guidelines for solving logarithmic equations}
\begin{itemize}
    \item Isolate the logarithmic term on one side of the equation. (Reduce the logarithms)
    \item Write the equation in exponental form.
    \item Use algebra to solve for the variable.
\end{itemize}

\subsubsection{Some notes}

\paragraph{Example 1} We are going to use the same procedure from example 2 into example 1.
First we reduce the expression using laws of logarithms,
\begin{align*}
    \log\qty(x^2+1) &= \log\qty(x-2) + \log\qty(x+3) \\
    \log\qty(x^2+1) &= \log\qty(\qty(x-2)\qty(x+3))
\end{align*} 
Since both sides cannot be reduce an further, we apply an exponential function base $10$ in both sides,
\begin{align*}
    10^{\log\qty(x^2+1)} &= 10^{\log\qty(\qty(x-2)\qty(x+3))},
\end{align*}
then, using the properties of the inverse functions $a^{\log_a(x)}=x$,
\begin{align*}
    x^2+1 &= (x-2)(x+3) \\
    x^2+1 &= x^2\cancelto{x}{-2x+3x}-6 \\
    \cancelto{0}{x^2-x^2}-x &= \cancelto{-7}{-6-1} \\
-x &= -7 \\
    x &= 7.
\end{align*}
Which is the same result the we previously computed.

\subsection{Exercises}

Solve the following equations,
\begin{enumerate}
    \begin{minipage}[c]{0.45\textwidth}
        \item $4+3\log(2x)=16$
        \item $\log(x+2) + \log(x-1) = 1$
    \end{minipage}
    \begin{minipage}[c]{0.45\textwidth}
        \item $\ln(x) = 10$
        \item $\log_2(x) + \log_2(x-3) = 1$
    \end{minipage}
\end{enumerate}

\end{document}
