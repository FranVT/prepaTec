\documentclass[../main.tex]{subfiles}

\begin{document}

\section{Review}

Now we are going to review all the topics for the partial exam.

\subsection{Exponential and logarithmic functions.}
Determine the asymptote, critical point, domain and range of the following functions,
\begin{enumerate}[label=\alph*]
    \begin{minipage}[c]{0.45\textwidth}
    \item $f(x) = -e^{x+2}-3$
    \item $f(x) = -\log_3\qty(x+4)+2$
    \end{minipage}
    \hfill
    \begin{minipage}[c]{0.45\textwidth}
    \item $f(x) = \ln\qty(-x+3)-1$
    \item $f(x) = \qty(5/3)^-x-4$
    \end{minipage}
\end{enumerate}

\subsection{Properties of logarithms}
Use the properties of logarithms to condense or expand the logarithmic expression.
\begin{enumerate}[label=\alph*]
    \begin{minipage}[c]{0.45\textwidth}
        \item $\frac{1}{5}\qty[\log_8\qty(y)+\frac{2}{3}\log_8\qty(y+4)]-4\log_8\qty(y-1)$
    \end{minipage}
    \hfill
    \begin{minipage}[c]{0.45\textwidth}
    \item $\log_2\qty[\frac{x^4}{w^5}\sqrt{\frac{y}{z^3}}]$
    \end{minipage}
\end{enumerate}

\subsection{Exponential and logarithmic equations}
Solve the following equations. 
Report the truncated value to 6 decimal places.
\begin{enumerate}[label=\alph*]
    \begin{minipage}[c]{0.45\textwidth}
    \item $\log\qty(3-x)-\log\qty(12-x)=-1$
    \item $e^{x^2+6} = e^{5x}$
    \item $15\qty(e^{x-1}) = 100$
    \end{minipage}
    \hfill
    \begin{minipage}[c]{0.45\textwidth}
    \item $6^{x-3} = 2^{x+2}$
    \item $\log_8\qty(x-6)+\log_8\qty(x+6) = 2$
    \item $2\log_4\qty(x) - \log_4\qty(x+5) = 3 + \log_4\qty(4)$
    \end{minipage}
\end{enumerate}

\subsection{Modeling exponential and logarithmic functions}

\paragraph{a)} a total of \$\num{6000} is invested at an annual interest of 4.3\%.
Write the function of this situation.
Find the final amount after 3 years if it is compounded annually, monthly and daily.

\paragraph{b)} Suppose that ypu invest \$\num{2000} at an annual interest rate of 15\%, compounded monthly.
How long will it take for your money to be \$\num{10000}?

\paragraph{c)} The population of a certain city is given by $P(t)=618e^{0.051t}$, where $P$ is measured in thousands of people and $t$ is measured in years from the year $2000$.
Find the population for the years 2010 and 2020.
Find when the population will be 3 millon.

\paragraph{d)} the number of bacteria inside a culture is modeled by the function $n(t) = 500e^{0.45t}$, where $t$ is measured in hours.
Which is the initial population and the growth rate (as a percentage) of this culture?
Wich will be the population of bateria after 3 hours?
How many hours after will the population of bacteria reach \num{10000} bacteria?

\paragraph{e)} The area covered by a certain population of bacteria increases according to a continuous exonential growth model.
Suppose that a sample culture has an initial area of \SI{76.6}{\milli\meter} and an observed tripling time of 23 days.
Let $t$ be the time in days passed and let $y$ be the area of the sample at time $t$.
what will the area of the sample be in t days?

\paragraph{f)} The world's population was 5.7 billion in 1995, and the annual growth rate was 2\%.
Write the model of this situation.
Which was the expected population in year 2015.
If the world's population in 2015 came up as 7 bilion, which is the real annual growth rate?

\paragraph{g)} Suppose that the number of milligrams of a certain drug that is in a patient's bloodstream for a specific number of hours after the drug is injected is given by $D(h)=40e^{-0.2h}.$
When the number of milligrams reaches 12, the drug is to be injected again.
How much time is needed between injections?

\paragraph{h)} a sample of a radioactive substance has a in initial mass of 763.3 mg.
this substance follows a continuous exponential decay model and has a half-life of 5 days.
How much will be present in 22 days?

\paragraph{i)} The half-life of Radium-226 is 1600 years.
Suppose that there is a sample of 22 mg.
Write the function that models the remaining mass after $t$ years.
How much mass will remain after 4000 years?
How much time time will it take for the sample to be only 18 mg?

\paragraph{j)} The remaining mass of a sample of 40 g. of thorium-234 after a certain amount of days is given by the function $m(t)=40e^{-0.0277t}$.
How much mass will be remaining after 60 days?
How much time will it take for the sample to be only 10 g.?
Calculate the half-life of Thorium-234.

\paragraph{k)} The sounf level $\beta$ in decibels, with an intensity $I$ is given by $\beta\qty(I) = 10\log\qty(I/I_o)$.
Where $I_o$ is \SI{d-12}{\watt\per\centi\meter\squared} of intensity, the finest detectable sound.
calculate the sound level $\beta$ for a pneumatic drill that operates to an intensity of $I=10^{-1}$.
Calculate the intensity fr the pain threshold of 120 dB.

\paragraph{l)} The magnitude of an eqarthequake on the Richter scale is calculated from the equation $R=\log\qty(I/I_o)$, where $I_o$ is the samllest earthquake that can be recorded with a seismograph.
Based on this information, how intentse is an earthquake of 9.2 in the Richter scale?
The intensity of an earthquake is \num{100000} times grater than the barely human-felt seismic motions (Magnitude 3).
Determine the magnitude on the Richter scale.

\paragraph{m)} Hydrogen potential (pH) is a number used to describe the acidity or basicity of a chemical substance and is defined by the equation $\mathrm{pH}=-\log\qty[H^+]$, where $H^+$ measures the concentration of hydrogen ions in moles per liter.
Determine the pH of a beer if $H^+=\num{6.4d-5}$.
The concentration of hydrogen ions in a substance having a pH of 3.4.

\paragraph{n)} The atmospheric pressure $P$ in milliliters of mercury (mm Hg) at an altitude $b$ is measured in kilometers.
From this information, calculate,
What is the value of the atmospheric pressure at 10 km above sea level?
At what altitude above sea level is the atmospheric pressure 370 mmHg?

\end{document}
