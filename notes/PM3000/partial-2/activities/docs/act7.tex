\documentclass[../main.tex]{subfiles}

\begin{document}

\section{Modeling with exponential functions: Compound Interest.}

In this session we are going to see another application of the exponential functions, the compound interest.
Lets assume that an ammount of money $P$ is invested at an interest rate $r$ per time period.
If we want to know the amount of money that the interest ``creates'' after that time period, we can computed by adding the initial amount with the initial amount times the interest,
\begin{align*}
    \$ &= P + Pr \\
       &= P\qty(1+r).
\end{align*}
Now, let say, that we re-invest the final ammount of money, $P\qty(1+r)$, for anothe period of time, that is,
\begin{align*}
    \$ &= P\qty(1+r) + P\qty(1+r)r \\
       &= P\qty(1+r)\qty(1+r) \\
       &= P\qty(1+r)^2.
\end{align*}
If we repeat the procedure a third time,
\begin{align*}
    \$ &= P\qty(1+r)^2 + P\qty(1+r)^2r \\
       &= P\qty(1+r)^2\qty(1+r) \\
       &= P\qty(1+r)^3.
\end{align*}
So we can get a general function that estimates the amount of money if we re-invest $k$ times the money,
\begin{align*}
    A = P\qty(1+i)^k.
\end{align*}

Now, we can think of the interest as an constant rate that is compounded $n$ times per year.
That is, that a rate $r$ is going to be applied during a period of time in small $i$ portions $n$ times.
For example, lets say that an annual interest rate $r$ is going to be applied during 6 months.
The monthly interest $i$ is going to be applied $k=6$ times in amounts of $r/6$.
Taking that into account we can generalize the function $A = P\qty(1+i)^k$ as follows,
\begin{definition}{Compound Interest}{lbl}
    \begin{gather*}
        A(t) = P\qty(1+\frac{r}{n})^{nt}.
    \end{gather*}
    Where $A(t)$ is the amount of money after $t$ years,
    $P$ is the initial amount of money,
    $r$ is the interest rate per year (each year we are going to reach that interest),
    $n$ is the number of times in which the rate is compounded per year,
    and $t$ is the number of years.
\end{definition}

Now, lets see and example,
\begin{example}{Time for an Investment to double}{label}
A sum of \$5000 is investead at an interest rate of 5\% per year.
Find the time required for the money to double if the interest is compounded semiannually (two times per year).

Now, we need to identify the parameters of the function, that is, $P$, $r$ and $n$.
Since the initial amount is \$5000, then $P=5000$. 
Also, we are expecting a \%5 increase each year, we know that $r=0.05$.
Finally, the interest rate ``executed'' twice a year, hence $n=2$.
With all that information we have the following function,
\begin{gather*}
    A(t) = 5000\qty(1+\frac{0.05}{2})^{2t}.
\end{gather*}

To find the time in which the initial amount is double, we propose the following equation,
\begin{gather*}
    2(5000) = 5000\qty(1+\frac{0.05}{2})^{2t}.
\end{gather*}
Now, we solve for $t$,
\begin{align*}
    2(5000)& = 5000\qty(1+\frac{0.05}{2})^{2t} \\
    \log\qty(2) &= \log\qty[\qty(1+\frac{0.05}{2})^{2t}] \\
    \log\qty(2) &= 2t\log\qty[\qty(1+\frac{0.05}{2})] \\
             2t &= \frac{\log\qty(2)}{\log\qty[\qty(1+\frac{0.05}{2})]} \\
             t &= \frac{\log\qty(2)}{2\log\qty[\qty(1+\frac{0.05}{2})]} \\
             t &\approx 14.035517
\end{align*}


\end{example}


\section{Exercises}

\paragraph{a)} Lets assume that a sum of \$1000 is invested at an interest rate of 12\% per year.
Find the amounts in the account after 3 years if the interes is compounde annually, semiannually, quarterly, monthly and daily.

\paragraph{b)} If \$\num{10000} is invested at an interest rate of 3\% per year, compounded semiannually, find the value of the investment after 5 years, 10 years and 15 years.

\paragraph{c)} Find the time in which the initial amount of money is tripled compounded monthly with an interest rate of 3.8 \%.


\end{document}
