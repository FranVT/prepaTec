\documentclass[../main.tex]{subfiles}

\begin{document}

\section{Exponential equations}

\subsection{Examples}

\paragraph{1} To solve the equation $5^{2x}=5^{x+1}$ we can use tha fact that is a bijective function, that is, is one to one.
Since is a bijective function and have the same base, we can equal the exponents,
\begin{gather*}
    2x=x+1.
\end{gather*}
Now, we can apply algebra tools to solve for $x$,
\begin{align*}
    \qty(2x)-x &= \qty(x+1)-x \\
    \cancelto{x}{2x-x} &= \cancelto{0}{x-x}+1 \\
    x &= 1.
\end{align*}
Hence, the answer is $x=1$.
To fact check tha answer, we can subsitute the value in the original equation,
\begin{align*}
    \left. 5^{2x}\right\bigg|_{x=1}=\left. 5^{x+1}\right\bigg|_{x=1} \\
    5^{2\qty(1)}=5^{1+1} \\
    5^{2}=5^{2}.
\end{align*}
Since both side of the equal sign are the same value we check that $x=1$ is a solution of the equation.

This example is a special case, so now we are going to explore an other example to see if we can get a general and robust methodology to solve this type of equations.

\paragraph{2} Consider the following equation $3^{x+2}=7$.
For this case we are going to use the Product law of logarithms $\log\qty(x^a)=a\log(x)$.
First, we apply the $\log$ function in both sides of the equation,
\begin{align*}
    3^{x+2} &= 7 \\
    \log\qty(3^{x+2}) &= \log\qty(7) \\
    \qty(x+2)\log\qty(3) &= \log\qty(7) \\
    \qty(x+2) &= \frac{\log\qty(7)}{\log\qty(3)} \\
    x + 2 &= \frac{\log\qty(7)}{\log\qty(3)} \\
    x &= \frac{\log\qty(7)}{\log\qty(3)} -2.
\end{align*}

Check the answer by computing the numeric value using your calculator.

\paragraph{Guidelines for solving Exponential equations} From the last example we can abstract the procedure to the following steps to solve any exponential equation,
\begin{itemize}
    \item Isolate the expnential expression on one side of the equation.
    \item Take de logarithm of each side, then use the Laws of logarithms to reduce the expression.
    \item Use algebra tools to solve for the variable.
\end{itemize}

\subsubsection{Some notes}

\paragraph{Example 1} We can repeat the procedure of the second example for the first example as follows,
\begin{align*}
    5^{2x} &= 5^{x+1} \\
    \log\qty(5^{2x}) &= \log\qty(5^{x+1}) \\
    \qty(2x)\log\qty(5) &= \qty(x+1)\log\qty(5) \\
    \frac{\qty(2x)}{\qty(x+1)} &= \cancelto{1}{\frac{\log\qty(5)}{\log\qty(5)}} \\
    \frac{\qty(2x)}{\qty(x+1)} &= 1 \\
    2x &= 1\qty(x+1) \\
    2x-x &= 1 \\
    x\cancelto{1}{\qty(2-1)} &= 1 \\
    x &= 1.
\end{align*}
As we can see, we got the same result as before.
Even though the procedure seems to be mor laborious, is much more robust than the one used in the first example.

\paragraph{Example 2} In this example we saw how we can use the logarithms properties to solve an exponential equation.
However we use the logarithmic function with base $10$ ($\log$).
Now we are going to explore to use the fact of inverse function between the logarithmic and exponential functions.
Recalling that $\log_a(a^x) = x$ and that $a^{\log_a(x)} = x$, we can solve the second example as follows,
\begin{align*}
    3^{x+2} &= 7 \\
    \log_3\qty(3^{x+2}) &= \log_3\qty(7) \\
    \qty(x+2)\log_3\qty(3) &= \log_3\qty(7), 
\end{align*}
analysing the factor $\log_3\qty(3)$, we can use the property of the inverse function, $\log_3\qty(3)=1$,
\begin{align*}
    \qty(x+2)\cancelto{1}{\log_3\qty(3)} &= \log_3\qty(7) \\
    x+2 &= \log_3\qty(7) \\
    x &= \log_3\qty(7) - 2.
\end{align*}
If we cant to compute the numeric value of that expression we can use the change of base formula,
\begin{align*}
    x &= \log_3\qty(7) - 2 \\
      &= \frac{\log(7)}{\log(3)} -2.
\end{align*}
Which is exactly the same value as before.

\subsection{Exerceises}
Solve the following equations,
\begin{enumerate}
    \item $8e^{2x} = 20$
    \item $e^{3-2x} = 4$
    \item $e^{2x}-e^{x}-6=0$
    \item $3xe^x + x^2e^x = 0 $
\end{enumerate}

Be careful with excercises 2 and 4.

\end{document}
