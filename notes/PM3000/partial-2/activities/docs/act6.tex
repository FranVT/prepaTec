\documentclass[../main.tex]{subfiles}

\begin{document}

\section{Critical points}

Now, we are going to practice the procedure to get the critical point of the exponential and the logarithmic functions.
As an intuitively way to known what represent the critical point, we can make an analogy of the functions with ``hills''.
The exponential functions grows \textit{very quickly} as $x$ increases, it is like climbing a hill that keeps getting steeper-there's no peak or flat spot.
They're always rising (if the base is greater than 1) or always falling (if the base than 1) but never level off completely.
On the other hand, the logarithmic functions grows \textit{slowly}, like walking up a hill that gradually flattens out, always increasing (for bases grater than 1) but never level off completely.
The critical point, for exponential functions, is when the hill starts to grow \textit{too quickly} or, for the logarithmic function, it is when starts to grow \textit{slowly}.
Now we are going to see the mathematical procedure to find the critical point for each function.

\begin{example}{Critical point of exponential functions}{log-fun-def}

To find the critical point of the exponential function, we need to equalize the argument of the function to $0$.
That is, for a function $f(x) = a^x$, we set, $x=0$, hence, $f(0)=a^{0}=1$ and the critical point is $(0,1)$.
Now, lets suppose, that the argument of the exponential function is another function, like this \[f(x)=a^{g(x)}.\]
As before, we equate the argument of the exponential function to $0$, $g(x)=0$ and then we solve for $x$.

\paragraph{(a)} Finding the critical point of $f(x)=e^{-(3x-4)}$.
From that function, we identify the argument, $-(3x-4)$.
Then, we equalized the argument to zero and solve for $x$,
\begin{align*}
    -(3x-4) &= 0 \\
    -3x+4 &= 0 \\
    x = \frac{4}{3}.
\end{align*}
Therefore, the critical point for $f(x)=e^{-(3x-4)}$ is $(4/3,1)$.

\paragraph{(b)} Now lets compute the critical point of $f(x) = \exp\qty[\log\qty(x^2)+2]+8$,
we known that the argument is $\log\qty(x^2)+2$, hence we do as follows,
\begin{align*}
    \log\qty(x^2)+2 &= 0 \\
    \log\qty(x^2) &= -2 \\
    10^{\log\qty(x^2)} &= 10^{-2} \\
    x^2 &= 10^{-2} \\
    \qty(x^2)^{1/2} &= \qty(10^{-2})^{1/2} \\
    x &= 10^{-1} = 0.1.\\
\end{align*}
Hence, the critical point of the function $f(x) = \exp\qty[\log\qty(x^2)+2]+8$ is $(0.1,9)$.

\end{example}

\begin{example}{Critical point of logarithmic functions}{log-fun-def}

To find the critical point of the logarithmic function, we need to equalize the argument of the function to $1$.
This is, for a function $f(x)=\log_a(x)$, we set $x=1$, hence, $f(1)=0$, leading to the critical point $(1,0)$.
Now, as before, let suppose that the argument of the logarithmic function is another function, \[f(x)=\log_a(g(x)).\]
Following the steps, we equalize the argument to zero, $g(x)=0$ and then, solve for $x$.

\paragraph{(a)} Now we are going to find the critical point of $f(x)=\log_3(x^2+2x+1)+4$.
The argument of the logarithmic function is a quadratic function\footnote{You can use any techinque which you feel more comfortable, quadratic formula, factorization or any other.}, hence,
\begin{align*}
    x^2+2x+1 &= 1 \\
    x^2+2x &= 0 \\
    x(x+2) &= 0.
\end{align*}
The two values that fullfill the equation is $x=0$ and $x=-2$.
Hence, we have two critical points, $(0,4)$ and $(-2,4)$.
It is important to notice that we have two critical point, because the argument of the logarithmic function is a quadratic function, which has two solutions.

\paragraph{(b)} For the last example lets try the following function $f(x)=\log_8(e^{1/x-2}-1)$.
Now, the argument of the logarithmic function is $e^{1/x-2}-1$,
\begin{align*}
    e^{1/x-2}-1 &= 1 \\
    e^{1/x-2} &= 2 \\
    \ln\qty[e^{1/x-2}] &= \ln\qty[2] \\
    \qty(\frac{1}{x}-2)\cancelto{1}{\ln\qty[e^1]} &= \ln\qty[2] \\
    \frac{1}{x}-2 &= \ln\qty[2] \\
    \frac{1}{x} &= \ln\qty[2] + 2\\
    x &= \frac{1}{\ln\qty[2] + 2}.
\end{align*}
Finally, the critical point of the function $f(x)=\log_8(e^{1/x-2}-1)$ is $(\frac{1}{\ln\qty[2] + 2},0)$

\end{example}

\section{Exercises}

Find the critical points of the following functions,
\begin{enumerate}
    \begin{minipage}[c]{0.45\textwidth}
        \item $f(x)=\num{5d3}e^{0.5 x}$
        \item $g(x)=-\log\qty[8x]$
        \item $h(x)=500e^{-0.2 x + 3}+150$
    \end{minipage}
    \begin{minipage}[c]{0.45\textwidth}
        \item $m(x)=\log\qty[\num{1d3}e^{10 x}+100]$
        \item $n(x)=3(x-3)e^{-5(x-3)}$
        \item $p(x)=3(x-5)\log\qty[-1/2(x-5)]$
    \end{minipage}
\end{enumerate}


\end{document}
