\documentclass[../main-exe.tex]{subfiles}

\begin{document}

\section{Examples}

\begin{example}{Factored equation of the circle.}{~}
    Identify the radius and center of the following circle,
    \begin{gather*}
        \qty(x+3)^2 + \qty(y+1)^2 = 9.
    \end{gather*}
    Recalling the definition of the circle, $\qty(x-h)^2 + \qty(y-k)^2 = r^2$, and comparing both expressions, we can make the following relations,
    \begin{align*}
        x+3 \longleftrightarrow x-h,\quad y+1 \longleftrightarrow y-k,\quad 9 \longleftrightarrow r^2.
    \end{align*}
    Therefore,
    \begin{align*}
        3 = -h,\quad & 1 = -k,\quad  9 = r^2 \\
        h = -3,\quad & k = -1,\quad  r = 9^{1/2} = 3.
    \end{align*}

    This tell's us that the circle origin is at $(-3,-1)$ with a radius of $r=3$.

\end{example}

\begin{example}{Expanded equation of the circle.}{~}
    Identify the radius and center of the following circle,
    \begin{gather*}
        x^2 + y^2 -8x +4y +4 = 0. 
    \end{gather*}

    Recalling the note, we can make the following comparsion as before,
    \begin{gather*}
        x^2 + y^2 -8x +4y +4 = 0\quad\longleftrightarrow\quad x^2 + y^2 -2hx -2ky + h^2 + k^2 -r^2 = 0.
    \end{gather*}
    giving us the following relations,
    \begin{align*}
        -8x = -2hx,\quad & +4y = -2ky, \\
        h = \frac{-8}{-2},\quad & k = \frac{4}{-2}, \\
        h = 4,\quad & k = -2.
    \end{align*}
    Hence, the origin of the circle is at $(4,-2)$.
    To find the radius, we start by substituting the values of $h$ and $k$ into the equation and comparing the final terms,
    \begin{align*}
        x^2 + y^2 -8x +4y +4 = 0\quad&\longleftrightarrow\quad x^2 + y^2 -2\qty(4)x -2\qty(-2)y + (4)^2 + (-2)^2 -r^2 = 0. \\
        x^2 + y^2 -8x +4y +4 = 0\quad&\longleftrightarrow\quad x^2 + y^2 -8x +4y + 16 + 4 -r^2 = 0,
    \end{align*}
    therefore
    \begin{align*}
        4 &= 16 + 4 -r^2 \\
        4 - 20 &= -r^2 \\
        r &= 16^{1/2} \\
        r &= 4.
    \end{align*}
    Finally, the radius of the circle is 4.

\end{example}



\section{Exercises}

\begin{itemize}
    \item Write the equation of the circle with center at $(6,-9)$ and radius $1/4$.
\end{itemize}

\end{document}
