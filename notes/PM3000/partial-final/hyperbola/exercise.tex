\documentclass[../main-exe.tex]{subfiles}

\begin{document}

\section{Examples}

\begin{example}{Factored form of a vertical hyperbola}{~}
    We are going to find the center, vertices, foci and asymptotes of the following hyperbola, \[\frac{(y-8)^2}{4}-\frac{(x+1)^2}{9}=1.\]

    Let's start with the center, which is $C=(-1,8)$.
    Now that we know that and also the values of $a=2$ and $b=3$, we can compute $c$ using the relation $b^2\equiv c^2 -a^2,\rightarrow c = \sqrt{9 + 4} = \sqrt{13}$.
    An important observation is that the term with $x$ is negative, which tells us that is a vertical hyperbola.

    With that information we can compute the vertex of the hyperbola as follows,
    \begin{align*}
        v_1 &= (-1,8-2) &\quad v_2 &= (-1,8+2) \\
        v_1 &= (-1,6) &\quad v_2 &= (-1,10)
    \end{align*}

    Then, we do something similar for the foci,
    \begin{align*}
        F_1 &= (-1,8-\sqrt{13}) &\quad F_2 &= (-1,8+\sqrt{13}) \\
        F_1 &= (-1,4.39) &\quad F_2 &= (-1,11.60)
    \end{align*}

    Finally, to compute the asymptotes of the hyperbola, we modify the general equation for hyperbola asymptotes since is a vertical hyperbola, $y-k=\pm\frac{a}{b}\qty(x-h)$,
    \begin{gather*}
        y-k=\pm\frac{a}{b}\qty(x-h)\rightarrow y-8=\pm\frac{2}{3}\qty(x+1) \\
        y = \pm\frac{2}{3}\qty(x+1) +8
    \end{gather*}
\end{example}

\begin{example}{Factored form of a horizontal hyperbola}{~}
    We are going to find the center, vertices, foci and asymptotes of the following hyperbola, \[\frac{(x+8)^2}{25}-\frac{(y-9)^2}{4}=1.\]

    Let's start with the center, which is $C=(-8,9)$.
    Now that we know that and also the values of $a=5$ and $b=2$, we can compute $c$ using the relation $b^2\equiv c^2 -a^2,\rightarrow c = \sqrt{25 + 4} = \sqrt{29}$.
    An important observation is that the term with $y$ is negative, which tells us that is a horizontal hyperbola.

    With that information we can compute the vertex of the hyperbola as follows,
    \begin{align*}
        v_1 &= (-8-5,9) &\quad v_2 &= (-8+5,9) \\
        v_1 &= (-13,9) &\quad v_2 &= (-3,9)
    \end{align*}

    Then, we do something similar for the foci,
    \begin{align*}
        F_1 &= (-8-\sqrt{29},9) &\quad F_2 &= (-8+\sqrt{29},9) \\
        F_1 &= (-13.38,9) &\quad F_2 &= (-2.61,9)
    \end{align*}

    Finally, to compute the asymptotes of the hyperbola, we use the general equation for hyperbola, $x-h=\pm\frac{a}{b}\qty(y-k)$,
    \begin{gather*}
        x-h=\pm\frac{a}{b}\qty(y-k)\rightarrow x+8=\pm\frac{5}{2}\qty(y-9) \\
        x = \pm\frac{5}{2}\qty(y-9) - 8
    \end{gather*}
\end{example}

\section{Exercises}

\subsection{Hyperbola}

\begin{itemize}
    \item Construct the hyperbola equation from it's asymptotes, $y = \pm 1/2(x+4)-3$.
    \item Find the center, vertices, foci and asymptotes of the following hyperbola $(x-3)^2/9-(y-6)^2/4=1$.
\end{itemize}

\subsection{Review}

\begin{itemize}
    \item Solve for $x$, $\log_5(4x)-3=-8$
    \item Determine the horizontal asymptote for the following function $f(x)=1/3(e^x-4)^4$
    \item Determine the vertical and horizontal asymptotes of the following function $f(x) = (7x-15)/(x-5)$.
    \item Find the inverse function of $f(x)=2x+4$.
\end{itemize}

\end{document}
