\documentclass[../main-exe.tex]{subfiles}

\begin{document}

\section{Examples}

\begin{example}{Factored form of the parabola}{}
We are going to write the factored form of the parabola with the following information, the focus is at $F=(-3,-7)$ and the directrix is at $y=3$.

Since the directrix is an horizontal line we are going to use the folowing expression for the parabola, $(x-h)^2=4a(y-k)$.
We know that $(h,k)$ is the coordinate for the vertex of the parabola and the distance from the vertex to the focus is the same distance from the vertex to the directrix and that distance is the parameter $a$.
Therefore we need to compute the distance in the $y$ direction from the focus to the directrixand divided by two,
\begin{align*}
    a &= \frac{\abs{F_y - L}}{2} \\
      &= \frac{\abs{-7-3}}{2} \\
      &= 5.
\end{align*}
Now, we add $5$ to the $y$ component of the focus, giving us that the vertex is at $V=(-3,-2)$.
Finally we substitute those values into the factored equation,
\begin{gather*}
    (x+3)^2=20(y+2).
\end{gather*}

\end{example}

\begin{example}{Expanded form of the parabola}{}
We are going to find the vertex, focus and directrix of the following parabola $3x^2+24x+y+47=0$.

As we did we the ellipse, we are going to use the same methodology to find the vertex, focus and directrix of the parabola in its expanded expression.

\end{example}

\section{Exercises}

\subsection{Parabola}

\begin{itemize}
    \item Find the vertex, focus and directrix of the parabola $-y^2+2y-x+1=0$.
    \item find the equation of the parabola with vertex $(2,1)$ and directrix $y=-5$
    \item Find the intersection points (2 points) of between these two parabolas, $x^2=4(3/2)y$ and $y^2=4(3/2)x$.
\end{itemize}

\subsection{Review}

\begin{itemize}
    \item Consider the following function $f(x)=3/2 x^2$ and $g(x)=4/3x$. Compute $f(x)/g(x)$ and determine the domain and range of the new function.
    \item Define the coordinate of the empty hole $f(x)=(x^3(x-3))/(4(x-3))$.
    \item Use the change base formula to compute $log_{1-5/3}(2000)$.
    \item An unkwon element has a half-life of 800 days. Suppose that you have a sample of $400$ mg. Which is the decay rate of this element?
\end{itemize}

\end{document}
