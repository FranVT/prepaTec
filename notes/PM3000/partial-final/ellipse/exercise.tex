\documentclass[../main-exe.tex]{subfiles}

\begin{document}

\section{Examples}

\begin{example}{Factored equation of the ellipse}{~}

Find the center, axis and foci of the following ellipse, 
\begin{gather*}
    \frac{\qty(x-2)^2}{100} + \frac{\qty(y-5)^2}{64} = 1.
\end{gather*}

Comparing the given equation with the general equation for the ellipse we can get the following insight,
\begin{gather*}
    \frac{\qty(x-h)^2}{a^2} + \frac{\qty(y-k)^2}{b^2} = 1\quad\longleftrightarrow\quad\frac{\qty(x-2)^2}{100} + \frac{\qty(y-5)^2}{64} = 1 \\
    -h = -2,\quad -k=-5,\quad a^2 = 100,\quad b^2 = 64 \\
    h = 2,\quad k=5, \quad a = 10,\quad b=8.
\end{gather*}

Therefore, we know that the orign of the ellipse is at $(2,5)$, also that the semiminor axis is equal to $8$ and the semimajor axis is equal to $10$.
Lastly, to compute the foci of the ellipse we need to add and substract the $x$ component of the origin to the semimajor axis,
\begin{gather*}
    F_1 = (2,5) - (10,0) = (-8,5),\qquad F_2 = (2,5) + (10,0) = (12,5)
\end{gather*}

\end{example}

\begin{example}{Expanded equation of the ellipse}{~}

Find the center, axis and foci of the following ellipse,
\begin{gather*}
    x^2 + 4y^2 + 6x + 16y +3 = 0.
\end{gather*}

As before, we compare this expression with the expanded form of the ellipse,
\begin{align*}
     \frac{1}{a^2}\qty(x^2-2xh+h^2) + \frac{1}{b^2}\qty(y^2-2yk+k^2) = 1
    &\longleftrightarrow
    \qty(x^2 + 6x) + 4\qty(y^2+ 4y) = -3.
\end{align*}

As we can see, we are missing terms.
If we just say that $a=1$ and $b=1/2$, there is no solution for the equation.
Hence, we need to complete the square ($(x+\alpha)^2 &= x^2 +2\alpha x + \alpha^2$),
\begin{align*}
    \implies 6x = 2\alpha x\rightarrow \alpha = 3,\quad & \implies 4y = 2\alpha y\rightarrow \alpha = 2
    \\
    (x+3)^2 = x^2 +6x + 9, &\quad (y+2)^2 = y^2 +4y + 4
\end{align*}
To keep the same expression we need to re-write those terms as follows,
\begin{gather*}
    \qty(x^2 + 6x) = (x+3)^2-9,\quad \qty(y^2+ 4y) = (y+2)^2 -4,
\end{gather*}
therefore,
\begin{align*}
    (x+3)^2-9+ 4\qty[(y+2)^2 -4] &= -3 \\
    (x+3)^2 + 4(y+2)^2 &= 22,
\end{align*}
finally, we divide by $22$ the equation,
\begin{align*}
    \frac{1}{22}(x+3)^2 + \frac{4}{22}(y+2)^2 = 1 &\longleftrightarrow\frac{\qty(x-h)^2}{a^2} + \frac{\qty(y-k)^2}{b^2} = 1 \\
    3 = -h,\quad 2 = -k, &\qquad\frac{1}{22} = a^2,\quad \frac{2}{11} = b^2 \\
    h = -3,\quad k = -2,&\qquad a = \sqrt{22},\quad  b = \sqrt{\frac{2}{11}}.
\end{align*}

Hence, the ellipse is center at $O=(-3,-2)$ with focal points at $F_1=(-3-\sqrt{22},-2)$ and $F_2=(-3+\sqrt{22},-2)$.

\end{example}


\section{Exercises}

\subsection{Ellipse}

\begin{itemize}
    \item Find the center, vertices and foci of the following ellipse \[\frac{\qty(x-1)^2}{16} + \frac{\qty(y+5)^2}{25} = 1\].
    \item Write the compacted equation of the ellipse $16x^2-48x+9y^2+12y+28=0$.
\end{itemize}

\subsection{Review}

\begin{itemize}
    \item The hydrogen ion concentration of a sample of a mystery liquid is $\mathrm{H}^+=\num{6.88d-3}$.
        Calculate the pH of the substance considering that $\mathrm{pH}=-\log\qty[\mathrm{H}^+]$.
    \item Determine if the following functions are inverses by computing both compositions, $f(x)=(x-5)/2$ and $g(x)=2x+5$.
    \item Express the domain and rage of the functions $f(x)=(x-5)^{1/2}$ and $g(x)=1/(x-2)^2$.
    \item Find the critical point of the function $f(x)=\exp\qty[\qty(x-4)\qty(x+4)].$
\end{itemize}

\end{document}
